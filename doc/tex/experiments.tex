\section{Evalutation}
\label{sec:experiments}

\subsection{Experimental Setup}
\subsubsection{System specifications}
\label{subsec:systemspecs}
For our experiments, we used a server with Intel i5-3470 with 3.20GHz speed. It has 4 sockets with 1 thread per socket. 
It has also 128 KiB L1d, 128 KiB L1i, 1MiB L2 and 6MiB L3 cache. Due to the CPU architecture, 
we decided in our experiments to use 1-2-4 cores/threads, 
to test parallelization, because it does not support more threads per CPU.

Regarding the memory, it has 8GB DDR3 RAM. Specifically, it has 2 dims of 4 GB each with a speed of 1333 MHz and a width of 64 bits. 
About storage, it has an SSD drive with 120 GB capacity.

\subsubsection{Query characteristics}
For benchmarking our DBS, all DBS will be evaluated in terms of performance execution of 
four queries. 
The query \textbf{311} consists of a \emph{DISTINCT} aggregate function and a 
Python UDF applied to the data. The \textbf{Weblogs} query consists mainly of a complex SELECT which 
applies most of the UDFs and a JOIN of the two datasets of the query. 
Lastly, \textbf{Zillow1} and \textbf{Zillow2} are two different queries for the same dataset. 
Zillow1 consists of nested SELECTs where the respective UDFs are applied,
and the Zillow2 query consists of many aggregate 
functions (SUM, COUNT, and GROUP BY).

\subsection{Setup DBS for experiments}
\subsubsection{PostgreSQL}
In order to run Python UDFs in PostgreSQL, we need to install the PL\/Python 
procedural language by running the command \emph{CREATE EXTENSION plpython
3u;} as a superuser in the PostgreSQL shell. 
This language allows us to write functions in Python and call them within
PostgreSQL queries. After this, we are able to run normal Python UDFs in
PostgreSQL. 
Because PostgreSQL is a row-based database, it feeds one row of data into 
each function call, so there is no need for any special changes to the 
Python code (such as the addition of iteration).

\subsubsection{MonetDB}
In order to run Python-implemented UDFs in MonetDB, we need to modify the 
configuration of the database using the command 
\emph{monetdb set embedded\_py=3 <dbname>}. 
It should be noted that this is done for each database individually within
the DBMS. Because MonetDB is a column-based database, each UDF accepts data
in the form of a table. 
Thus, we need to make a loop to pass through all the data. In traditional
row-based databases, this is not necessary, as it is done automatically by the respective DBMS.


\subsubsection{Vertica}
Python scalar UDFs in Vertica were implemented by following the steps below. 
\begin{enumerate}
	\item Install \emph{vertica\_sdk} library in the .py file.
	\item Write your Python function: The Python function should take 
	one or more inputs and return a single value, using two classes, namely
	function's class, fuction's factory class.
	\item Compile the Python function: The \emph{vertica\_sdk} library provides 
	a udf decorator that you can use to compile your Python function into a shared library.
	\item Load the shared library into Vertica: You can use the 
	\emph{CREATE OR REPLACE LIBRARY} statement to load the shared library into Vertica, 
	specifying at the end the programming language (i.e \emph{LANGUAGE Python}).
	The Library name and the path of the .py file must be specified as part of this statement.
	\item Create all python functions by using the command \emph{CREATE OR REPLACE FUNCTION}
	followed by UDF's name, programming language, name of factory class and library~\ref{lst:verticaUDF} .
\end{enumerate}
An example of a library can be found 
\href{https://www.vertica.com/docs/9.2.x/HTML/Content/Authoring/ExtendingVertica/UDx/ScalarFunctions/Python/PythonExampleAdd2Ints.htm}{here}.
\begin{lstlisting}[language=sh, caption={Example of UDF in Vertica},label={lst:verticaUDF}] 
	CREATE OR REPLACE FUNCTION extract_ip 
	AS LANGUAGE 'Python' NAME 'extract_ip_factory' 
	LIBRARY weblogslib fenced;
\end{lstlisting}

\subsubsection{MongoDB}
MongoDB is a NoSQL database and does not support executing Python functions
natively. 
However, one can execute JavaScript functions within MongoDB using the 
\emph {function} 
method, which allows one to run JavaScript code directly on the database server.
So our first alternative to run a Python function in MongoDB was to
translate it into JavaScript and then run it using \emph{function}. 
Nevertheless, that way has some limitations and is generally discouraged in production environments, 
as it can have security implications and can block the server for other operations while it's running. 
If one need to run complex logic in MongoDB, 
it's usually better to implement it outside of the database and interact with MongoDB 
using the appropriate drivers and APIs.
We used PyMongo, the Python driver for MongoDB~\footnote{\url{https://api.mongodb.com/python/current/}},
to interact with a MongoDB database from one's Python code. 
With PyMongo, one can execute operations like reading, writing, and updating documents, 
creating and managing collections, and running aggregation pipelines.


\subsection{Results}

In this section, we will present the experiments we conducted for each DBMS and analyze them. 
We chose to examine the behavior of systems with warm and cold cache memory and with the use of parallelism. 
Particularly in MongoDB, we deemed it worth comparing the performance of the Database in running UDFs using 
Pandas in Python and natively in MongoEngine (using regular expressions).

We also need to mention that measurements were made with different data volumes, 
from x1 to x128 (comparing cache until x32) from the initial size of the data, 
doubling each time the data (i.e., we have x1, x2, x4, ..., x128). 
In the following charts to better present the results and more clearly, we chose not to place all the measurements 
(however, all measurements are available for checking). Finally, each measurement is the average of 5 reputations.

\subsubsection{Cache}
\label{subsec:cache}
Reading data from disk is still much slower than reading from memories. In
order to maintain the overall speed of the system, memory is used to cache the most
recent read data, so next time when the same data is requested; the system will read
directly from memory instead of disks~\cite{Smit82}. 
To check the caching effects of the systems, every query in each DBS executed in both warm and cold cache. 
The reason for this experiment is to observe the performance of the storage systems for the environments 
with an initial state or limited memory that do not utilize the caching effect fully. 
First, cold cache is an environment where data and indexes do not reside in the memory, i.e., queries have
never been executed. 
Second, warm cache is an environment where data and indexes
reside in the memory by executing the same query prior to conducting the experiment. 
For cold cache we dropped the file systems caches using
\begin{lstlisting}[language=bash, caption= {UNIX Command for clean cache}, label={lst:paralDF}]
	echo 3 > sudo /proc/sys/vm/drop_caches
\end{lstlisting} 
before each query is performed on OS level. 
This command frees page caches in the Linux system. 

We should not expect any DBS processes to show large memory use, due to the significant small size of our datasets;
our biggest file consists of approximately 6,400,000 records, corresponds to 1.36GB. Eventually, 
warm and cold cache differences are almost insignificant. 

% You should not expect postgres processes to show large memory use, even if the whole database is cached in RAM.
% That is because PostgreSQL relies on buffered reads from the operating system buffer cache. 
% In simplified terms, when PostgreSQL does a read(), the OS looks to see whether the requested blocks are 
% cached in the "free" RAM that it uses for disk cache. If the block is in cache, the OS returns it almost instantly. 
% If the block is not in cache the OS reads it from disk, adds it to the disk cache, and returns the block.
% Subsequent reads will fetch it from the cache unless it's displaced from the cache by other blocks.

\begin{figure}[!ht]
\begin{tikzpicture}
\begin{axis}[
	symbolic x coords={x1, x16, x32},
	enlarge x limits=0.25,
	legend style={at={(0.25,0.96)}, anchor=north east}
]
\addplot+ [bar width=0.3cm, ybar, color1!80!black, fill=color1, text=black] coordinates {(x1, 0.0894) (x16, 0.953) (x32, 1.8502)};
\addplot+ [bar width=0.3cm, ybar, color2!80!black, fill=color2, text=black] coordinates {(x1, 0.163) (x16, 0.995) (x32, 1.8902)};
\addplot+ [bar width=0.3cm, ybar, color3!80!black, fill=color3, text=black] coordinates {(x1, 0.0907822) (x16, 1.1996344) (x32, 2.3687442)};
\addplot+ [bar width=0.3cm, ybar, color4!80!black, fill=color4, text=black] coordinates {(x1, 0.0934848) (x16, 1.2112494) (x32, 2.3839544)};
\addplot+ [bar width=0.3cm, ybar, color5!80!black, fill=color5, text=black] coordinates {(x1, 0.3174) (x16, 4.2282) (x32, 8.3862)};
\addplot+ [bar width=0.3cm, ybar, color6!80!black, fill=color6, text=black] coordinates {(x1, 0.3136) (x16, 4.1932) (x32, 8.2184)};
\addplot+ [bar width=0.3cm, ybar, color7!80!black, fill=color7, text=black] coordinates {(x1, 0.3408) (x16, 2.5616) (x32, 4.8236)};
\addplot+ [bar width=0.3cm, ybar, color8!80!black, fill=color8, text=black] coordinates {(x1, 0.3514) (x16, 2.5368) (x32, 4.8488)};
\legend{MonetDB cold, MonetDB warm, Mongo cold, Mongo warm, PostgreSQL cold, PostgreSQL warm, Vertica cold, Vertica warm}
\end{axis}
\end{tikzpicture}
	\caption{311 (Cold/Warm cache)}
	\label{fig:311Cache}

\begin{tikzpicture}
\begin{axis}[
	symbolic x coords={x1, x16, x32},
	enlarge x limits=0.25,
	legend style={at={(0.25,0.96)}, anchor=north east}
]
\addplot+ [bar width=0.3cm, ybar, color1!80!black, fill=color1, text=black] coordinates {(x1, 0.0916) (x16, 1.2374) (x32, 2.3126)};
\addplot+ [bar width=0.3cm, ybar, color2!80!black, fill=color2, text=black] coordinates {(x1, 0.1456) (x16, 1.1384) (x32, 2.2192)};
\addplot+ [bar width=0.3cm, ybar, color3!80!black, fill=color3, text=black] coordinates {(x1, 0.677313) (x16, 11.2072838) (x32, 22.3317652)};
\addplot+ [bar width=0.3cm, ybar, color4!80!black, fill=color4, text=black] coordinates {(x1, 0.6868898) (x16, 11.283492) (x32, 22.2838622)};
\addplot+ [bar width=0.3cm, ybar, color5!80!black, fill=color5, text=black] coordinates {(x1, 0.2538) (x16, 3.44) (x32, 6.7694)};
\addplot+ [bar width=0.3cm, ybar, color6!80!black, fill=color6, text=black] coordinates {(x1, 0.2516) (x16, 3.532) (x32, 7.0788)};
\addplot+ [bar width=0.3cm, ybar, color7!80!black, fill=color7, text=black] coordinates {(x1, 1.4508) (x16, 3.7988) (x32, 6.0396)};
\addplot+ [bar width=0.3cm, ybar, color8!80!black, fill=color8, text=black] coordinates {(x1, 1.4228) (x16, 3.7426) (x32, 6.3762)};
\legend{MonetDB cold, MonetDB warm, Mongo cold, Mongo warm, PostgreSQL cold, PostgreSQL warm, Vertica cold, Vertica warm}
\end{axis}
\end{tikzpicture}
	\caption{Weblogs (Cold/Warm cache)}
	\label{fig:WeblogsCache}
\end{figure}

\begin{figure}[!ht]
\begin{tikzpicture}
\begin{axis}[
	symbolic x coords={x1, x16, x32},
	enlarge x limits=0.25,
	legend style={at={(0.25,0.96)}, anchor=north east}
]
\addplot+ [bar width=0.3cm, ybar, color1!80!black, fill=color1, text=black] coordinates {(x1, 0.3462) (x16, 3.7882) (x32, 7.4704)};
\addplot+ [bar width=0.3cm, ybar, color2!80!black, fill=color2, text=black] coordinates {(x1, 0.3582) (x16, 3.8824) (x32, 7.6168)};
\addplot+ [bar width=0.3cm, ybar, color3!80!black, fill=color3, text=black] coordinates {(x1, 3.1553452) (x16, 53.32206) (x32, 108.055524)};
\addplot+ [bar width=0.3cm, ybar, color4!80!black, fill=color4, text=black] coordinates {(x1, 3.2449956) (x16, 52.5604554) (x32, 107.6443716)};
\addplot+ [bar width=0.3cm, ybar, color5!80!black, fill=color5, text=black] coordinates {(x1, 1.2028) (x16, 17.9406) (x32, 34.9936)};
\addplot+ [bar width=0.3cm, ybar, color6!80!black, fill=color6, text=black] coordinates {(x1, 1.1228) (x16, 16.4402) (x32, 32.262)};
\addplot+ [bar width=0.3cm, ybar, color7!80!black, fill=color7, text=black] coordinates {(x1, 0.6754) (x16, 0.7314) (x32, 0.7644)};
\addplot+ [bar width=0.3cm, ybar, color8!80!black, fill=color8, text=black] coordinates {(x1, 0.6742) (x16, 0.7704) (x32, 0.765)};
\legend{MonetDB cold, MonetDB warm, Mongo cold, Mongo warm, PostgreSQL cold, PostgreSQL warm, Vertica cold, Vertica warm}
\end{axis}
\end{tikzpicture}
	\caption{Zillow 1 (Cold/Warm cache)}
	\label{fig:Zillow1Cache}

\begin{tikzpicture}
\begin{axis}[
	symbolic x coords={x1, x16, x32},
	enlarge x limits=0.25,
	legend style={at={(0.25,0.96)}, anchor=north east}
]
\addplot+ [bar width=0.3cm, ybar, color1!80!black, fill=color1, text=black] coordinates {(x1, 0.5458) (x16, 7.6736) (x32, 15.477)};
\addplot+ [bar width=0.3cm, ybar, color2!80!black, fill=color2, text=black] coordinates {(x1, 0.5362) (x16, 7.6634) (x32, 15.167)};
\addplot+ [bar width=0.3cm, ybar, color3!80!black, fill=color3, text=black] coordinates {(x1, 3.7440264) (x16, 61.4363946) (x32, 122.2530776)};
\addplot+ [bar width=0.3cm, ybar, color4!80!black, fill=color4, text=black] coordinates {(x1, 3.7367452) (x16, 61.312647) (x32, 124.5240324)};
\addplot+ [bar width=0.3cm, ybar, color5!80!black, fill=color5, text=black] coordinates {(x1, 1.7184) (x16, 26.139) (x32, 51.1124)};
\addplot+ [bar width=0.3cm, ybar, color6!80!black, fill=color6, text=black] coordinates {(x1, 1.6854) (x16, 25.5984) (x32, 50.4318)};
\addplot+ [bar width=0.3cm, ybar, color7!80!black, fill=color7, text=black] coordinates {(x1, 2.419) (x16, 14.8176) (x32, 28.8634)};
\addplot+ [bar width=0.3cm, ybar, color8!80!black, fill=color8, text=black] coordinates {(x1, 2.4372) (x16, 14.9642) (x32, 28.7596)};
\legend{MonetDB cold, MonetDB warm, Mongo cold, Mongo warm, PostgreSQL cold, PostgreSQL warm, Vertica cold, Vertica warm}
\end{axis}
\end{tikzpicture}
	\caption{Zillow 2 (Cold/Warm cache)}
	\label{fig:Zillow2Cache}
\end{figure}

\begin{figure}[!ht]
	\begin{tikzpicture}
		\begin{axis}[
			symbolic x coords={x8, x16, x32, x128},
			enlarge x limits=0.25,
			legend style={at={(0.17,0.96)}, anchor=north east}
		]
		\addplot+ [bar width=0.3cm, ybar, color1!80!black, fill=color1, text=black] coordinates {(x8, 15.162) (x16, 28.787) (x32, 54.982) (x128, 210.854)};
		\addplot+ [bar width=0.3cm, ybar, color2!80!black, fill=color2, text=black] coordinates {(x8, 12.666) (x16, 26.030) (x32, 51.167) (x128, 202.331)};
		\addplot+ [bar width=0.3cm, ybar, color3!80!black, fill=color3, text=black] coordinates {(x8, 12.656) (x16, 26.012) (x32, 51.311) (x128, 201.930)};
		\addplot+ [bar width=0.3cm, ybar, color4!80!black, fill=color4, text=black] coordinates {(x8, 12.650) (x16, 25.995) (x32, 51.141) (x128, 202.150)};
		\addplot+ [bar width=0.3cm, ybar, color5!80!black, fill=color5, text=black] coordinates {(x8, 12.631) (x16, 25.939) (x32, 51.236) (x128, 202.274)};
		\legend{1st run, 2nd run, 3rd run, 4th run, 5th run}
		\end{axis}
		\end{tikzpicture}
			\caption{PostgreSQL - Zillow 2 (HDD Cache)}
			\label{fig:PostgresHDDCache}
\end{figure}


\subsubsection{Parallelization}

Starting off, we should mention again that the system we ran our experiments on consists of a 4-core processor with 1 thread per core. 
Thus, in our experiments to explore parallelism, we looked at differences between 1, 2, and 4 processes and threads. 
We chose this approach because we deemed it meaningless to use more processes/threads, as it would lead to the throttling of our machine. 
It would have been of interest to test the experiments on larger and more powerful machines and see if any differences arose 
and to what extent they are explained by the increased machine power, or to what extent the performance-to-implementation-cost ratio is satisfactory 
to make such changes.

Additionally, we consider it important to note that Python, for security reasons, allows only one thread to hold control of the Python interpreter. 
This is known as the Python Global Interpreter Lock or GIL~\cite{PythonGIL}. This is a mechanism in CPython, 
the default and most widely used implementation of the 
Python programming language, to ensure that only one thread executes Python bytecodes at a time. 
This lock is necessary because CPython's memory management is not thread-safe, meaning only one thread can be in a state of execution at any time. 
The impact of the GIL is not noticeable for developers running single-threaded programs, 
but it can be a performance bottleneck in CPU-bound and multi-threaded code. Suppose we want to write a program that needs to use multiple threads 
for maximum performance. In that case, we may consider using an alternative implementation of Python, such as Jython or IronPython, 
which do not have a GIL or use multiple processes instead of threads.

\paragraph{PostgreSQL} implements parallelism at the process level. 
This means that it could theoretically utilize the capabilities of Python 
for parallelism (as we mentioned at the beginning of the chapter) 
since Python also implements parallelism at the process level. The
parameters that we need to take into consideration in order to 
implement\/activate parallelism are:
\begin{enumerate}
	\item $max\_worker\_processes$: the maximum number of background 
	processes that the system can support.
	\item $max\_parallel\_workers\_per\_gather$: the maximum number of 
	workers that can be started by a single Gather or Gather Merge node
	\item $max\_parallel\_workers$: the maximum number of workers that the 
	system can support for parallel queries
	\item $max\_parallel\_maintenance\_workers$: the maximum number of 
	parallel workers that a single utility command can start
\end{enumerate}
Also, the $dynamic\_shared\_memory\_type$ parameter should be enabled with
the appropriate value, depending on the system that the DBMS is running on.

Below, we see the diagrams that show the results we had running the queries 
on 1, 2, and 4 cores.

\begin{figure}[!ht]
	\begin{minipage}{.5\textwidth}
	\begin{tikzpicture}
	\begin{axis}[
		symbolic x coords={x1, x32, x128},
		enlarge x limits=0.18,
		legend style={at={(0.64,0.93)}, anchor=north east},
		width = 7cm,
		height = 7cm
	]
	\addplot+ [bar width=0.3cm, ybar, color1!80!black, fill=color1, text=black] coordinates {(x1, 0.3176) (x32, 8.5206) (x128, 32.5684) };
	\addplot+ [bar width=0.3cm, ybar, color2!80!black, fill=color2, text=black] coordinates {(x1, 0.3198) (x32, 8.5554) (x128, 31.7244) };
	\addplot+ [bar width=0.3cm, ybar, color3!80!black, fill=color3, text=black] coordinates {(x1, 0.356) (x32, 8.7202) (x128, 32.5684) };
	\legend{PostgreSQL 1 process, PostgreSQL 1 process, PostgreSQL 1 process}
	\end{axis}
	\end{tikzpicture}
		\caption{PostgreSQL - 311 (Parallelism)}
		\label{fig:311ParallelismPostgreSQL}
	\end{minipage}%
	\begin{minipage}{.5\textwidth}
	\begin{tikzpicture}
	\begin{axis}[
		symbolic x coords={x1, x32, x128},
		enlarge x limits=0.18,
		legend style={at={(0.64,0.93)}, anchor=north east},
		width = 7cm,
		height = 7cm
	]
	\addplot+ [bar width=0.3cm, ybar, color1!80!black, fill=color1, text=black] coordinates {(x1, 0.2494) (x32, 6.6864) (x128, 25.4232) };
	\addplot+ [bar width=0.3cm, ybar, color2!80!black, fill=color2, text=black] coordinates {(x1, 0.2476) (x32, 6.7178) (x128, 26.367) };
	\addplot+ [bar width=0.3cm, ybar, color3!80!black, fill=color3, text=black] coordinates {(x1, 0.2432) (x32, 6.6638) (x128, 25.8148) };
	\legend{PostgreSQL 1 process, PostgreSQL 1 process, PostgreSQL 1 process}
	\end{axis}
	\end{tikzpicture}
		\caption{PostgreSQL - Weblogs (Parallelism)}
		\label{fig:WeblogsParallelismPostgreSQL}
	\end{minipage}
	\end{figure}
	\begin{figure}[!ht]
	\begin{minipage}{.5\textwidth}
	\begin{tikzpicture}
	\begin{axis}[
		symbolic x coords={x1, x32, x128},
		enlarge x limits=0.18,
		legend style={at={(0.64,0.93)}, anchor=north east},
		width = 7cm,
		height = 7cm
	]
	\addplot+ [bar width=0.3cm, ybar, color1!80!black, fill=color1, text=black] coordinates {(x1, 1.1708) (x32, 33.6204) (x128, 132.1814) };
	\addplot+ [bar width=0.3cm, ybar, color2!80!black, fill=color2, text=black] coordinates {(x1, 1.1578) (x32, 33.4664) (x128, 132.8554) };
	\addplot+ [bar width=0.3cm, ybar, color3!80!black, fill=color3, text=black] coordinates {(x1, 1.1584) (x32, 32.577) (x128, 133.2398) };
	\legend{PostgreSQL 1 process, PostgreSQL 2 process, PostgreSQL 4 process}
	\end{axis}
	\end{tikzpicture}
		\caption{PostgreSQL - Zillow1 (Parallelism)}
		\label{fig:Zillow1ParallelismPostgreSQL}
	\end{minipage}%
	\begin{minipage}{.5\textwidth}
	\begin{tikzpicture}
	\begin{axis}[
		symbolic x coords={x1, x32, x128},
		enlarge x limits=0.18,
		legend style={at={(0.64,0.93)}, anchor=north east},
		width = 7cm,
		height = 7cm
	]
	\addplot+ [bar width=0.3cm, ybar, color1!80!black, fill=color1, text=black] coordinates {(x1, 1.7136) (x32, 51.5364) (x128, 202.3072) };
	\addplot+ [bar width=0.3cm, ybar, color2!80!black, fill=color2, text=black] coordinates {(x1, 1.6276) (x32, 49.7454) (x128, 206.8864) };
	\addplot+ [bar width=0.3cm, ybar, color3!80!black, fill=color3, text=black] coordinates {(x1, 1.6186) (x32, 49.3152) (x128, 200.7306) };
	\legend{PostgreSQL 1 process, PostgreSQL 2 process, PostgreSQL 4 process}
	\end{axis}
	\end{tikzpicture}
		\caption{PostgreSQL - Zillow2 (Parallelism)}
		\label{fig:Zillow2ParallelismPostgreSQL}
	\end{minipage}
	\end{figure}	

\paragraph{MonetDB}
\label{par:MonetFBPar}
on the other hand, implements parallelism at thread 
level. 
That is, for the same process, it generates many threads that operate on 
the same memory, having common global variables, heap, and code. 
This is in contrast to what Python provides us, where UDFs are implemented. 
As we will see in the charts, MonetDB cannot leverage parallelism enough to
give us satisfactory results. 
Also, in the charts it shows that we tried parallelism at both thread level 
and database-pipeline level.

Regarding the thread-level parallelization, we modified the number of 
nthreads through the database configuration, using the 
\emph{monetdb set nthreads=x <dbname>} command. 
As for the pipeline method mentioned earlier, MonetDB internally implements
a type of pipeline to give a speedup in processing the data. 
The user has the option to disable it with the 
\emph{monetdb set optpipe=sequential\_pipe <dbname>} command. 
We thought it would be interesting to see how such an option affects the 
speed of our results. 
In the following charts, in the sequential pipeline, we used 4 threads 
which is the default option of MonetDB.

\begin{figure}[t]
	\begin{minipage}{.5\textwidth}
	\begin{tikzpicture}
	\begin{axis}[
		symbolic x coords={x1, x32, x128},
		enlarge x limits=0.25,
		legend style={at={(0.51,0.95)}, anchor=north east},
		width = 7cm,
		height = 7cm
	]
	\addplot+ [bar width=0.3cm, ybar, color1!80!black, fill=color1, text=black] coordinates {(x1, 0.161) (x32, 1.9026) (x128, 7.4264) };
	\addplot+ [bar width=0.3cm, ybar, color2!80!black, fill=color2, text=black] coordinates {(x1, 0.1622) (x32, 1.8896) (x128, 7.3956) };
	\addplot+ [bar width=0.3cm, ybar, color3!80!black, fill=color3, text=black] coordinates {(x1, 0.163) (x32, 1.8902) (x128, 7.417) };
	\addplot+ [bar width=0.3cm, ybar, color4!80!black, fill=color4, text=black] coordinates {(x1, 0.1608) (x32, 1.8808) (x128, 7.3984) };
	\legend{MonetDB 1 thread, MonetDB 2 thread, MonetDB 4 thread, MonetDB Seq Pipe}
	\end{axis}
	\end{tikzpicture}
		\caption{MonetDB - 311 (Parallelism)}
		\label{fig:311ParallelismMonet}
	\end{minipage}%
	\begin{minipage}{.5\textwidth}
	\begin{tikzpicture}
	\begin{axis}[
		symbolic x coords={x1, x32, x128},
		enlarge x limits=0.25,
		legend style={at={(0.51,0.95)}, anchor=north east},
		width = 7cm,
		height = 7cm
	]
	\addplot+ [bar width=0.3cm, ybar, color1!80!black, fill=color1, text=black] coordinates {(x1, 0.146) (x32, 2.2292) (x128, 9.328) };
	\addplot+ [bar width=0.3cm, ybar, color2!80!black, fill=color2, text=black] coordinates {(x1, 0.1464) (x32, 2.2208) (x128, 9.6558) };
	\addplot+ [bar width=0.3cm, ybar, color3!80!black, fill=color3, text=black] coordinates {(x1, 0.1456) (x32, 2.2192) (x128, 9.4814) };
	\addplot+ [bar width=0.3cm, ybar, color4!80!black, fill=color4, text=black] coordinates {(x1, 0.1464) (x32, 2.1796) (x128, 9.1778) };
	\legend{MonetDB 1 thread, MonetDB 2 thread, MonetDB 4 thread, MonetDB Seq Pipe}
	\end{axis}
	\end{tikzpicture}
		\caption{MonetDB - Weblogs (Parallelism)}
		\label{fig:WeblogsParallelismMonet}
	\end{minipage}
	\end{figure}
	\begin{figure}[t]
	\begin{minipage}{.5\textwidth}
	\begin{tikzpicture}
	\begin{axis}[
		symbolic x coords={x1, x32, x128},
		enlarge x limits=0.25,
		legend style={at={(0.51,0.95)}, anchor=north east},
		width = 7cm,
		height = 7cm
	]
	\addplot+ [bar width=0.3cm, ybar, color1!80!black, fill=color1, text=black] coordinates {(x1, 0.3454) (x32, 7.9144) (x128, 31.263) };
	\addplot+ [bar width=0.3cm, ybar, color2!80!black, fill=color2, text=black] coordinates {(x1, 0.3622) (x32, 7.6228) (x128, 30.8102) };
	\addplot+ [bar width=0.3cm, ybar, color3!80!black, fill=color3, text=black] coordinates {(x1, 0.3582) (x32, 7.6168) (x128, 30.754) };
	\addplot+ [bar width=0.3cm, ybar, color4!80!black, fill=color4, text=black] coordinates {(x1, 0.339) (x32, 7.6994) (x128, 30.8414) };
	\legend{MonetDB 1 thread, MonetDB 2 thread, MonetDB 4 thread, MonetDB Seq Pipe}
	\end{axis}
	\end{tikzpicture}
		\caption{MonetDB - Zillow1 (Parallelism)}
		\label{fig:Zillow1ParallelismMonet}
	\end{minipage}%
	\begin{minipage}{.5\textwidth}
	\begin{tikzpicture}
	\begin{axis}[
		symbolic x coords={x1, x32, x128},
		enlarge x limits=0.25,
		legend style={at={(0.51,0.95)}, anchor=north east},
		width = 7cm,
		height = 7cm
	]
	\addplot+ [bar width=0.3cm, ybar, color1!80!black, fill=color1, text=black] coordinates {(x1, 0.5304) (x32, 15.1888) (x128, 62.946) };
	\addplot+ [bar width=0.3cm, ybar, color2!80!black, fill=color2, text=black] coordinates {(x1, 0.5506) (x32, 15.2562) (x128, 62.6158) };
	\addplot+ [bar width=0.3cm, ybar, color3!80!black, fill=color3, text=black] coordinates {(x1, 0.5362) (x32, 15.167) (x128, 63.015) };
	\addplot+ [bar width=0.3cm, ybar, color4!80!black, fill=color4, text=black] coordinates {(x1, 0.539) (x32, 15.1864) (x128, 62.7628) };
	\legend{MonetDB 1 thread, MonetDB 2 thread, MonetDB 4 thread, MonetDB Seq Pipe}
	\end{axis}
	\end{tikzpicture}
		\caption{MonetDB - Zillow2 (Parallelism)}
		\label{fig:Zillow2ParallelismMonet}
	\end{minipage}
	\end{figure}

\paragraph{Vertica}
unlike PostgreSQL and Monet, which do not employ parallelism in the queries tested, Vertica adopts a different 
approach utilizing parallelism at both the node and cluster levels~\cite{LFVT12}.

The Vertica Execution Engine (EE) is responsible for executing all processes for the query plan. 
A query plan is represented as a tree of operations, where each operator performs a specific algorithm. 
The EE uses a multi-threaded and pipelined architecture, allowing for multiple operators to run 
concurrently and multiple threads to execute code for a single operator. Additionally, the EE is 
vectorized and requests blocks of rows instead of individual rows, similar to the C-Store database.

At the single node level, Vertica employs parallelism on a smaller scale by dividing the data locally 
and processing it in parallel to keep all cores fully utilized. For instance, multiple GroupBy operators 
run in parallel and request data from the StorageUnion~\footnote{StorageUnion is a feature in the Vertica 
database management system that provides support for data storage and retrieval across different storage media.}.

At the cluster level, Vertica divides the data across multiple nodes and executes the query in parallel 
on each node. The nodes exchange intermediate results and aggregate the final results to produce the 
final answer, resulting in a "massively parallel processing" (MPP) architecture. This architecture enables 
Vertica to efficiently handle large data volumes and complex queries in a scalable manner. In the experiments 
conducted, we created three resource pools as depicted in table~\ref{tab:vertBenchProfiles}. 
Further information regarding the resource pools can be found in Vertica 
documentation.~\footnote{\url{https://www.vertica.com/docs/12.0.x/HTML/Content/Authoring/SQLReferenceManual/SystemTables/CATALOG/RESOURCE_POOLS.htm}}

There is a possibility that we could see differences in all queries with larger data. 
As we see in Figure~\ref{fig:Zillow1ParallelismVertica}, the difference increases when we increase the volume. 
It should be noted that parallelism does not always bring better results. 
This is most evident in Figure~\ref{fig:Zillow1ParallelismVertica} at x128 where Profile 4 takes longer than 2.

\begin{table}[!ht]
    \caption{Vertica Benchmark Profiles}
    \label{tab:vertBenchProfiles}
    \centering
	\scalebox{0.7}{
		\begin{tabular}{|l|c|c|c|}
			\hline
			\textbf{Property} & \textbf{Profile 1} & \textbf{Profile 2} & \textbf{Profile 4} \\ \hline
			\emph{EXECUTIONPARALLELISM} & 1 & 2 & 4 \\ \hline
			\emph{MAXCONCURRENCY} & 1 & 2 & 4 \\ \hline
			\emph{MAXMEMORYSIZE} & 100\% & 100\% & 100\% \\ \hline
			\emph{MAXQUERYMEMORYSIZE} & 100\% & 100\% & 100\% \\ \hline
			\emph{PRIORITY} & 100 & 100 & 100 \\ \hline
			\emph{RUNTIMEPRIORITY} & HIGH & HIGH & HIGH \\ \hline
			\emph{RUNTIMEPRIORITYTHRESHOLD} & 0 & 0 & 0 \\ \hline
			\emph{QUEUETIMEOUT} & NONE & NONE & NONE \\ \hline
			\emph{RUNTIMECAP} & NONE & NONE & NONE \\ \hline
		\end{tabular}}
\end{table}

\begin{figure}[!ht]
	\begin{minipage}{.5\textwidth}
	\begin{tikzpicture}
	\begin{axis}[
		symbolic x coords={x1, x16, x32},
		enlarge x limits=0.18,
		legend style={at={(0.4,0.95)}, anchor=north east},
		width = 7cm,
		height = 7cm
	]
	\addplot+ [bar width=0.3cm, ybar, color1!80!black, fill=color1, text=black] coordinates {(x1, 0.3334) (x16, 2.5006) (x32, 4.7736) };
	\addplot+ [bar width=0.3cm, ybar, color2!80!black, fill=color2, text=black] coordinates {(x1, 0.35) (x16, 2.4584) (x32, 4.7226) };
	\addplot+ [bar width=0.3cm, ybar, color3!80!black, fill=color3, text=black] coordinates {(x1, 0.3272) (x16, 2.4998) (x32, 4.7522) };
	\legend{Vertica P1, Vertica P2, Vertica P4}
	\end{axis}
	\end{tikzpicture}
		\caption{Vertica - 311 (Parallelism)}
		\label{fig:311ParallelismVertica}
	\end{minipage}%
	\begin{minipage}{.5\textwidth}
	\begin{tikzpicture}
	\begin{axis}[
		symbolic x coords={x1, x16, x32},
		enlarge x limits=0.18,
		legend style={at={(0.4,0.95)}, anchor=north east},
		width = 7cm,
		height = 7cm
	]
	\addplot+ [bar width=0.3cm, ybar, color1!80!black, fill=color1, text=black] coordinates {(x1, 1.3702) (x16, 3.855) (x32, 6.3892) };
	\addplot+ [bar width=0.3cm, ybar, color2!80!black, fill=color2, text=black] coordinates {(x1, 1.3554) (x16, 3.9134) (x32, 6.5084) };
	\addplot+ [bar width=0.3cm, ybar, color3!80!black, fill=color3, text=black] coordinates {(x1, 1.3434) (x16, 3.6574) (x32, 6.1112) };
	\legend{Vertica P1, Vertica P2, Vertica P4}
	\end{axis}
	\end{tikzpicture}
		\caption{Vertica - Weblogs (Parallelism)}
		\label{fig:WeblogsParallelismVertica}
	\end{minipage}
	\end{figure}
	\begin{figure}[!ht]
	\begin{minipage}{.5\textwidth}
	\begin{tikzpicture}
	\begin{axis}[
		symbolic x coords={x1, x32, x128},
		enlarge x limits=0.18,
		enlarge y limits=5.7,
		ymin=0.7,
		legend style={at={(0.4,0.95)}, anchor=north east},
		width = 7cm,
		height = 7cm
	]
	\addplot+ [bar width=0.3cm, ybar, color1!80!black, fill=color1, text=black] coordinates {(x1, 0.7416) (x32, 0.656) (x128, 0.7076) };
	\addplot+ [bar width=0.3cm, ybar, color2!80!black, fill=color2, text=black] coordinates {(x1, 0.6944) (x32, 0.7126) (x128, 0.7512) };
	\addplot+ [bar width=0.3cm, ybar, color3!80!black, fill=color3, text=black] coordinates {(x1, 0.6816) (x32, 0.714) (x128, 0.823) };
	\legend{Vertica P1, Vertica P2, Vertica P4}
	\end{axis}
	\end{tikzpicture}
		\caption{Vertica - Zillow1 (Parallelism)}
		\label{fig:Zillow1ParallelismVertica}
	\end{minipage}%
	\begin{minipage}{.5\textwidth}
	\begin{tikzpicture}
	\begin{axis}[
		symbolic x coords={x1, x32, x128},
		enlarge x limits=0.18,
		legend style={at={(0.4,0.95)}, anchor=north east},
		width = 7cm,
		height = 7cm
	]
	\addplot+ [bar width=0.3cm, ybar, color1!80!black, fill=color1, text=black] coordinates {(x1, 2.0954) (x32, 41.0024) (x128, 161.8368) };
	\addplot+ [bar width=0.3cm, ybar, color2!80!black, fill=color2, text=black] coordinates {(x1, 2.257) (x32, 28.473) (x128, 111.4684) };
	\addplot+ [bar width=0.3cm, ybar, color3!80!black, fill=color3, text=black] coordinates {(x1, 2.048) (x32, 28.8494) (x128, 115.4764) };
	\legend{Vertica P1, Vertica P2, Vertica P4}
	\end{axis}
	\end{tikzpicture}
		\caption{Vertica - Zillow2 (Parallelism)}
		\label{fig:Zillow2ParallelismVertica}
	\end{minipage}
	\end{figure}

\paragraph{MongoDB} in its native implementation, is capable 
of carrying query execution in parallel. 
Based on its original documentation~\cite{MogoDBConc}, mongod process uses a modified 
reader/writer lock with dynamic yielding on page faults
and long operations. Any number of concurrent 
read operations are allowed, but a write operation can block all 
other operation. 


can be the right choice for concurrency, 
as multi-granularity locking is used, which allows
operations to lock at the global, database or collection level. 
Additionally, for individual storage engines, it is allowed
to implement their own concurrency control below the collection level. 
In terms of compliance, pandas library was used, and the parallelism was done
on python level. In other words, we solve the problem of having 
a big dataframe, and one wants to apply a complex function to every 
record of this dataframe, which takes a lot of time. 
More specifically, we use a function~\ref{lst:paralDF} 
as proposed in this article~\cite{ParallelDF}, 
which breaks the dataframe into \emph{n\_cores} parts, 
and spawns \emph{n\_cores} processes which apply 
the function to all the pieces. Once 
Pool from multiprocessing 
library was used.
\begin{lstlisting}[language=Python, caption= {Kernel Parallelization in Python}, 
	label={lst:paralDF}]
	def parallelize_dataframe(df, func, n_cores=4):
		df_split = np.array_split(df, n_cores)
		pool = Pool(n_cores)
		df = pd.concat(pool.map(func, df_split))
		pool.close()
		pool.join()
		return df
\end{lstlisting}

	
\begin{figure}[t]
\begin{minipage}{.5\textwidth}
\begin{tikzpicture}
\begin{axis}[
	symbolic x coords={x1, x16, x32},
	enlarge x limits=0.18,
	legend style={at={(0.47,0.95)}, anchor=north east},
	width = 7cm,
	height = 7cm
]
\addplot+ [bar width=0.3cm, ybar, color1!80!black, fill=color1, text=black] coordinates {(x1, 0.0934848) (x16, 1.2112494) (x32, 2.3839544) };
\addplot+ [bar width=0.3cm, ybar, color2!80!black, fill=color2, text=black] coordinates {(x1, 0.1075664) (x16, 1.2260764) (x32, 2.4011928) };
\addplot+ [bar width=0.3cm, ybar, color3!80!black, fill=color3, text=black] coordinates {(x1, 0.113517) (x16, 1.215507) (x32, 2.3902328) };
\legend{Mongo 1 Core, Mongo 2 Core, Mongo 4 Core}
\end{axis}
\end{tikzpicture}
	\caption{Mongo - 311 (Parallelism)}
	\label{fig:311ParallelismMongo}
\end{minipage}%
\begin{minipage}{.5\textwidth}
\begin{tikzpicture}
\begin{axis}[
	symbolic x coords={x1, x16, x32},
	enlarge x limits=0.18,
	legend style={at={(0.47,0.95)}, anchor=north east},
	width = 7cm,
	height = 7cm
]
\addplot+ [bar width=0.3cm, ybar, color1!80!black, fill=color1, text=black] coordinates {(x1, 0.6939822) (x16, 11.303007) (x32, 22.288301) };
\addplot+ [bar width=0.3cm, ybar, color2!80!black, fill=color2, text=black] coordinates {(x1, 0.564085) (x16, 9.0017266) (x32, 17.6031448) };
\addplot+ [bar width=0.3cm, ybar, color3!80!black, fill=color3, text=black] coordinates {(x1, 0.48111) (x16, 7.5024536) (x32, 14.4992794) };
\legend{Mongo 1 Core, Mongo 2 Core, Mongo 4 Core}
\end{axis}
\end{tikzpicture}
	\caption{Mongo - Weblogs (Parallelism)}
	\label{fig:WeblogsParallelismMongo}
\end{minipage}
\end{figure}
\begin{figure}[t]
\begin{minipage}{.5\textwidth}
\begin{tikzpicture}
\begin{axis}[
	symbolic x coords={x1, x16, x32},
	enlarge x limits=0.18,
	legend style={at={(0.47,0.95)}, anchor=north east},
	width = 7cm,
	height = 7cm
]
\addplot+ [bar width=0.3cm, ybar, color1!80!black, fill=color1, text=black] coordinates {(x1, 3.2449956) (x16, 52.5604554) (x32, 107.6443716) };
\addplot+ [bar width=0.3cm, ybar, color2!80!black, fill=color2, text=black] coordinates {(x1, 2.3943826) (x16, 38.790527) (x32, 100.0086542) };
\addplot+ [bar width=0.3cm, ybar, color3!80!black, fill=color3, text=black] coordinates {(x1, 1.5410532) (x16, 25.9415926) (x32, 79.4014554) };
\legend{Mongo 1 Core, Mongo 2 Core, Mongo 4 Core}
\end{axis}
\end{tikzpicture}
	\caption{Mongo - Zillow1 (Parallelism)}
	\label{fig:Zillow1ParallelismMongo}
\end{minipage}%
\begin{minipage}{.5\textwidth}
\begin{tikzpicture}
\begin{axis}[
	symbolic x coords={x1, x16, x32},
	enlarge x limits=0.18,
	legend style={at={(0.47,0.95)}, anchor=north east},
	width = 7cm,
	height = 7cm
]
\addplot+ [bar width=0.3cm, ybar, color1!80!black, fill=color1, text=black] coordinates {(x1, 3.7367452) (x16, 61.8848998) (x32, 124.1408488) };
\addplot+ [bar width=0.3cm, ybar, color2!80!black, fill=color2, text=black] coordinates {(x1, 2.6603464) (x16, 43.2579538) (x32, 110.1501846) };
\addplot+ [bar width=0.3cm, ybar, color3!80!black, fill=color3, text=black] coordinates {(x1, 1.7257306) (x16, 29.2115762) (x32, 76.596979) };
\legend{Mongo 1 Core, Mongo 2 Core, Mongo 4 Core}
\end{axis}
\end{tikzpicture}
	\caption{Mongo - Zillow2 (Parallelism)}
	\label{fig:Zillow2ParallelismMongo}
\end{minipage}
\end{figure}