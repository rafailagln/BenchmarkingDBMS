\section{Mongo DB}
\label{sec:mongo}

MongoDB is a popular, open-source NoSQL database management system. 
It was first released in 2009 and was created with the goal of providing a database system 
that could handle the rapidly growing amounts of data being generated by modern web applications. 
MongoDB is a \emph{document-oriented database}, not a relational one. The primary reason
for moving away from the relational model is to make scaling out easier, but there are
some other advantages as well~\cite{HPMH13}.

A document-oriented database replaces the concept of a “row” with a more flexible
model, the “document.” By allowing embedded documents and arrays, the document-oriented approach 
makes it possible to represent complex hierarchical relationships
with a single record. This fits naturally into the way developers in modern object-oriented 
languages think about their data~\cite{HPMH13}.
There are also no predefined schemas: a document’s keys and values are not of fixed
types or sizes. Without a fixed schema, adding or removing fields as needed becomes
easier. Generally, this makes development faster as developers can quickly iterate. It is
also easier to experiment. Developers can try dozens of models for the data and then
choose the best one to pursue.

MongoDB is intended to be a general-purpose database, so aside from creating, reading,
updating, and deleting data, it provides an ever-growing list of unique features:
\begin{itemize}
    \item Indexing: generic secondary indexes, allowing a variety of fast queries,
    and provides unique, compound, geospatial, and full-text indexing capabilities as
    well
    \item Aggregation: \emph{aggregation pipeline} that allows you to build complex
    aggregations from simple pieces and allow the database to optimize it.
    \item Special collection types: time-to-live collections for data that should expire at a certain
    time, such as sessions. It also fixed-size collections, which are useful for
    holding recent data, such as logs.
    \item File storage: easy-to-use protocol for storing large files and file metadata
\end{itemize}


MongoDB is powerful but easy to get started with. Some 
of the basic concepts of MongoDB:
\begin{itemize}
    \item A document is the basic unit of data for MongoDB and is roughly equivalent to a
    row in a relational database management system (but much more expressive).
    \item Similarly, a collection can be thought of as a table with a dynamic schema.
    \item A single instance of MongoDB can host multiple independent databases, each of
    which can have its own collections.
    \item Every document has a special key, "\_id", that is unique within a collection.
    \item MongoDB comes with a simple but powerful JavaScript shell, which is useful for
    the administration of MongoDB instances and data manipulation.
\end{itemize}
