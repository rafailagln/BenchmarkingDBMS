\section{Vertica}
\label{sec:vertica}

HP Vertica was explicitly designed for analytic workloads rather than for transactional workloads. HP
Vertica’s free versions (HP Vertica Community Edition) are the Enterprise Edition (ver. 6.0) and Analytics Platform
(currently ver. 9.1) available for download.
% in the form of a virtual machine (https://www.vertica.com/log-in/) on which operating system CentOS (64bit version) is preinstalled.
The Vertica DB physically organises table data into projections, which are sorted subsets of the attributes of a
table. In practice, most applications have one super projection (containing every column of the anchoring
table) and zero to three narrow, non-super projections. Each projection has a specific sort order on which the data
is totally sorted. HP Vertica uses Read Optimized Store (ROS) and Write Optimized Store (WOS). Data in the ROS
is physically stored in multiple ROS containers on a standard file system. Each ROS container logically
contains some number of complete tuples sorted by the projection's sort order, stored as a pair of files per column.
HP Vertica is a true column store - column files may be independently retrieved as the storage is physically
separate. By dividing each on-disk structure into logical regions at runtime and processing the regions in parallel,
HP Vertica achieves high performance. 