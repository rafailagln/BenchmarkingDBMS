\section{Vertica}
\label{sec:vertica}

In 2005, Mike Stonebraker and colleagues published a paper
outlining a formal database system, called C-store.~\cite{SABC05} 
Stonebraker formed a company "Vertica" to build a commercial 
implementation of C-Store. Vertica was acquired by HP in 2011.
% C-Store and Vertica became pioneers of a new wave of systems that 
% departed significantly from the traditional RDBMS architecture.
% Vertica became the poster child for a “NewSQL” database.
%Subsequent to the release of the C-Store model, several other significant column-based systems entered the market, including InfoBright, VectorWise, and MonetDB. 
HP Vertica was explicitly designed for analytic workloads rather than
for transactional workloads. 
HP Vertica’s free versions (HP Vertica Community Edition) are the Enterprise Edition (ver. 6.0) and Analytics Platform (currently ver. 12.0) available for download.
% in the form of a virtual machine (https://www.vertica.com/log-in/) on which operating system CentOS (64bit version) is preinstalled.
The Vertica DB physically organises table data into projections, which are sorted subsets of the attributes of a table. 
More specifically, most applications have one super projection and zero to three narrow, non-super projections. 
Each projection has a specific sort order on which the data 
is totally sorted. 
HP Vertica uses Read Optimized Store (ROS) and Write Optimized Store (WOS). 
Data in the ROS is physically stored in multiple ROS containers on a standard file system. 
Each ROS container logically contains some number of complete tuples sorted by the projection's sort order, stored as a pair of files per column. 
HP Vertica is a true column store - column files may be independently retrieved as the storage is physically separate. 
By dividing each on-disk structure into logical regions at runtime and processing the regions in parallel, HP Vertica achieves high performance.~\cite{Harr15}