\section{Introduction}
\label{sec:intro}

% Necessity of such comparison 
Comparing different database management systems (DBMS) is essential in today's fast-paced technological world. 
With the increasing demand for data storage, retrieval, and management, it's crucial to choose the right DBMS 
that fits your specific business requirements. A comparison of different DBMSs can help you determine which 
one provides the best performance, scalability, security, and cost-effectiveness. 
This ensures that you are able to store and manage your data effectively, enabling you to make informed 
business decisions. 
Furthermore, as technology advances, new DBMSs emerge, and older systems become obsolete, 
regular comparison and evaluation are necessary to keep your business up-to-date with the latest 
technology.

% Definition of UDF
As the most popular language for querying relational data, SQL provides a convenient and efficient
mechanism for interacting with data stored in relational databases. Beyond providing the familiar
relational algebra operators, SQL also allows programmers to perform computations on data through 
SQL's highly optimized built-in functions as well as through \textit{user-defined functions} (UDFs)~\cite{ZC08}.


% Summary table for 4DBS
\begin{table}[!ht]
    \caption{Information for DBMS}
    \label{tab:alldbs}
    \centering
    \resizebox{\textwidth}{!}{
    \begin{tabular}{|l|c|c|c|c|}
    \hline
    \textbf{}                    & \textbf{MonetDB}       & \textbf{MongoDB}                                                                                       & \textbf{PostgreSQL}                                                            & \textbf{Vertica}                                                                                               \\ \hline
    \textbf{DB model}        & Relational DBMS        & Document store                                                                                         & Relational DBMS                                                                & Relational DBMS                                                                                                \\ \hline
    \textbf{Developer}               & MonetDB BV             & MongoDB, Inc                                                                                           & \begin{tabular}[c]{@{}c@{}}PostgreSQL Global \\ Development Group\end{tabular} & Micro Focus                                                                                                    \\ \hline
    \textbf{Initial release}         & 2004                   & 2009                                                                                                   & 1989                                                                           & 2005                                                                                                           \\ \hline
    \textbf{Current release}         & Jul2021-SP2, July 2021 & 6.0.1, August 2022                                                                                     & 15.1, November 2022                                                            & 11.1, February 2022                                                                                            \\ \hline
    \textbf{License}                 & Open Source            & Open Source                                                                                            & Open Source                                                                    & commercial                                                                                                     \\ \hline
    \textbf{Implemented} & C                      & C++                                                                                                    & C                                                                              & C++                                                                                                            \\ \hline
    \textbf{Data scheme}             & yes                    & schema-free                                                                                            & yes                                                                            & \begin{tabular}[c]{@{}c@{}}yes, but unstructured \\ data can be stored in \\ specific Flex-Tables\end{tabular} \\ \hline
    \textbf{Transaction}    & ACID                   & \begin{tabular}[c]{@{}c@{}}Multi-document ACID \\ Transactions with \\ snapshot isolation\end{tabular} & ACID                                                                           & ACID                                                                                                           \\ \hline
    \textbf{Concurrency}             & yes                    & yes                                                                                                    & yes                                                                            & yes                                                                                                            \\ \hline
    \textbf{Durability}              & yes                    & yes                                                                                                    & yes                                                                            & yes                                                                                                            \\ \hline
    \end{tabular}}
\end{table}

% Scope of assignment: Utilize python UDFs
This work aims to study and benchmark four widely-used big data systems based on their performance in handling
complex queries. More specifically, the four systems that were selected for the benchmarking are MonetDB, 
Postgres, Vertica, and MongoDB. An analytical table of some basic information for each DBMS is presented
in table~\ref{tab:alldbs}. The evaluation is based on their performance on running queries, which include UDFs.


% Difference of SQL and noSQL
\subsubsection{SQL vs NOSQL DBMS}
SQL (Structured Query Language) and NoSQL (Not Only SQL) are two types of database management systems that are widely used today. 
SQL databases are relational databases that store data in tables with defined relationships and adhere to a strict structure. 
On the other hand, NoSQL databases are non-relational databases that store data in various formats including documents, 
key-value pairs, and graph structures~\cite{M0M14}.
SQL databases are best suited for structured data with well-defined relationships, such as financial data and customer information. 
They are highly efficient for complex queries and transactions, and offer robust security features. 
On the other hand, NoSQL databases are best suited for handling large amounts of unstructured or semi-structured data, 
such as social media posts and online customer reviews. 
They are highly scalable, making them ideal for big data applications.

The choice between SQL and NoSQL depends on the specific requirements of your project and the nature of your data.
 While SQL databases offer reliability and security, NoSQL databases provide scalability and flexibility. 
 Both have their own strengths and weaknesses, and the decision of which to use should be based on the specific needs of your application.

 % Difference of row Vs. column based
\subsubsection{Row vs Column based databases}
Column-oriented database systems (column-stores) have attracted
a lot of attention in the past few years. Column-stores, in a
nutshell, store each database table column separately, with
attribute values belonging to the same column stored
contiguously, compressed, and densely packed, as opposed to
traditional database systems that store entire records (rows) one
after the other. Reading a subset of a table's columns becomes
faster, at the potential expense of excessive disk-head seeking
from column to column for scattered reads or updates~\cite{ABH09}.

Column-based databases have several advantages and disadvantages. 
One of the main advantages is improved storage efficiency and reduced disk I/O, 
as column-based databases only store the data needed for a particular query, 
rather than reading entire rows~\cite{BHSB14}. 
This results in faster query performance and reduced data processing times. 
Column-based databases are also well-suited for large scale data analysis and can handle massive amounts of data with ease. 
However, column-based databases can be challenging to implement and maintain, as they require specialized software and hardware. 
Additionally, they are not as flexible as row-based databases, and updating a single value can require rewriting an entire column. 
While column-based databases provide excellent performance for analytical queries, 
they may not be the best choice for transactional systems 
where fast write speeds and the ability to update individual values are essential.

Correspondingly, row-based databases have other pros and cons. 
Starting with their flexibility, as row-based databases allow for easy updates of individual values and fast write speeds.
This makes them ideal for transactional systems where frequent updates and real-time data retrieval is essential. 
Row-based databases also tend to be easier to implement and maintain, 
as they use a more familiar and straightforward data storage model. 
Nevertheless, row-based databases can be less efficient than column-based databases when it comes 
to large scale data analysis. 
This is because row-based databases must read entire rows of data, 
even if only a small portion of the data is needed for a particular query. 
This can result in slower query performance and increased data processing times, 
especially for large datasets. 
Lastly, row-based databases may not be as scalable as column-based databases, 
and can struggle to handle massive amounts of data~\cite{BHSB14}.

% Structure of this report 

The report begins by providing background information 
for every system, as well as, an overview 
of their main features and characteristics is provided.
The benchmarking process is then presented in detail; 
including the specific UDFs that were deployed in each system, 
the performance metrics that were collected, 
and the experimental setup that was used. 
The results are presented and analyzed, 
in terms of query execution time, 
based on data overload and resource utilization. 

% The report concludes by summarizing the main findings of the study and offering recommendations for data scientists and engineers who are looking to select a big data system for their data science projects. The report also highlights the limitations of the study and suggests potential areas for future research. Overall, the report aims to provide a comprehensive and objective comparison of the performance of the four big data systems and to help data scientists and engineers make informed decisions about which system is best suited for their specific data science needs.