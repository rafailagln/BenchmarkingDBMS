\section{MonetDB}
\label{sec:monet}


MonetDB is a powerful, open-source relational database management system that is optimized for analytical workloads, data mining, 
geographic information system (GIS), Resource Description Framework (RDF), text retrieval and sequence alignment processing. 
It uses a column-store approach, which is a different storage method than traditional row-based databases. 
This allows MonetDB to handle large amounts of data with high performance and efficiency. 
MonetDB is written in C, and it supports SQL as query language. It is originally developed at the Centrum Wiskunde \& Informatica (CWI) in the Netherlands.
MonetDB is particularly suitable for data warehousing, business intelligence, and scientific research and it is scalable and can handle large datasets. 
MonetDB also supports multiple languages, including Python, making it easy to integrate with other tools and technologies.

Python support means that the user can implement their own functions in the base using Python. 
It can then execute the queries appropriately by calling them as if they were built-in MonetDB functions.

Because MonetDB is a column-based database, each function takes the data as a pandas array and thus is taken into each UDF. For this reason, in the implementation of the UDF we iterate for each value of the table, so as to go through all the data. As we will see below (in PostgreSQL) traditional databases that are row-based do not have this problem.